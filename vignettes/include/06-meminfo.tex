\section{Hardware Memory Information: meminfo()}

\subsection{Using meminfo()}

As of \pkg{memuse}~$\geq 2.0$, we include the \code{meminfo()} utility.  This function gathers some minimal hardware information about available memory: physical and (when possible) virtual.  This is \emph{extremely} platform dependent, and we will discuss some of the specifics in the following subsection.  

To use, simply call:
\begin{lstlisting}
meminfo()
\end{lstlisting}

And a list will be returned containing some information.  By default, this will only contain a simplified profile of total and ``available'' (more on this later) physical memory.  For Linux, Mac, and Windows, the option is available for swap (or paging in the case of Windows) information.  




\subsection{Platform Bullshit}

The platform with the best support in the \pkg{memuse} package is Linux.  This is largely because getting this information in Linux is very simple --- on top of the data itself is very well-defined --- compared to other platforms.  So get your shit together, other OS devs.

For clarity, we define some terms here.

Swap is pretend ram that you probably shouldn't ever use; generally I would rather have a program crash than send me into a 40 hour swap death.

All operating systems basically treat swap/page files the same.  There are minor differences in allocation; for example, Mac OS X bizarrely uses unused space on the \code{/boot} partition for swap.  But the basic principle is the same: dump stuff to disk.

For the remainder, we will be discussing how various values for ram are calculated.



\subsubsection{Linux}
In Linux, the total available ram can be broken down into
\begin{itemize}
  \item truly unused (freeram)
  \item used for buffer cache (bufferram)
  \item used for file caching (cachedram)
  \item used by programs
\end{itemize}

By default, memuse will collapse the first three into \code{freeram}, because that's how most people think about it; and if you open up some kind of system monitor program, they do the same thing.  By \emph{truly unused}, we mean that nothing on the system has made any claim to it (according to the kernel\dots).  The memory allocated for buffers is .

For more details, \href{https://www.redhat.com/advice/tips/meminfo.html}{Redhat} has a pretty good breakdown of the various items.


These values can easily be found in \code{/proc/meminfo}, with a small sample of the file provided:

\begin{center}
\begin{minipage}{.4\textwidth}
\begin{Output}
MemTotal:        8053556 kB
MemFree:         1761144 kB
Buffers:          808096 kB
Cached:          2327648 kB
SwapCached:            8 kB
\end{Output}
\end{minipage}
\end{center}

The actual implementation of \code{meminfo()} uses \code{sysinfo} (see \code{man 2 sysinfo}) for everything but cached memory, which is read directly from \code{/proc/meminfo}.




\subsubsection{Windows}

The \href{http://msdn.microsoft.com/en-us/library/windows/desktop/aa366589%28v=vs.85%29.aspx}{Windows documentation} 
on this is surprisingly clear and thorough.  I would compliment them, except that the function returns \emph{nonzero upon success}, which is in violation of the \proglang{C} standard, not to mention common decency.  Good job, guys.

Also, this may crash your R session, so watch out.





\subsubsection{Mac OS X}






\subsubsection{FreeBSD}

For some reason, FreeBSD doesn't support \code{_SC_AVPHYS_PAGES} in \code{sysconf} like Solaris and Linux do.  So for getting free memory, I have to make a call to \code{sysctlbyname}.

Also, the only way I could see to support getting free swap on FreeBSD is to make libkvm a dependency, which is kind of ridiculous.  Also, I figure you folks will just call sysctl(8) anyway and wonder why I even went to all this trouble in the first place, so you get what you get.



\subsubsection{Miscellaneous *NIX Variants}

Everybody seems to have their own little wiggle to \code{sysctl} for getting various memory statistics, and I just don't have the kind of time or resources to set up 100 different BSD-variant vm's to test that the man pages aren't lying to me.  As such, all *NIX's outside of Linux and Mac get only total ram and ``free ram'', as reported by \code{sysconf}.

Contributions are welcome.

