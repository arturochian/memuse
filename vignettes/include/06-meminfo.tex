\section{Hardware Memory Information}

As of \thispackage version 2.0, some basic hardware memory information is 
available:

\begin{center}
\vspace{0.2cm}
\begin{tabular}{ll} \hline\hline
Function & Description of Measurement \\ \hline
\code{Sys.meminfo()} & Total and free RAM (also buffers and cache if 
available)\\
\code{Sys.swapinfo()} & Total and free swap/page space.\\
\code{Sys.procmem()} & Total and peak RAM usage by current \R process.\\
\code{Sys.cacheinfo()} & Cache sizes \\
\hline\hline
\end{tabular}
\vspace{0.2cm}
\end{center}

Not all platforms are supported, and even among those that are, not all 
measurements are available across all platforms.  The platform with the best 
support in the \pkg{memuse} package is Linux.  This 
is largely because getting this information in Linux is very simple --- on top 
of the data itself being very well-defined --- compared to other platforms.  So 
get your shit together, other OS devs.

All platforms get these values differently, and calculate them in slightly 
different ways.  About the only measurement that means the same thing on all 
platforms is ``total physical ram''.  Everything else is what you get when no 
one wants to standardize anything.

Not only do the return values themselves sometimes have different meanings on 
different platforms, but some things are/aren't returned on different platforms 
as well.  The remainder of this section will outline those differences, as a 
kind of angrily written reference standard.

The remainder of this section is dedicated to the sisyphean tragedy of 
multiplatform software development.




\subsection{Terminology}

For clarity, we define some terms here.

RAM is ``fast'' storage on your computer (if you pretend that cache doesn't 
exist, anyway).  \emph{Free ram} is the portion of ram not being used, either 
by the OS (when this information is possible to get) or by programs, hackers, 
etc.

Swap is pretend-ram that you shouldn't ever actually use.  I would much
rather have a program crash than send me into a 40 hour swap death.  In 
short, it's ``ram'' on your disk.  How swap is used and/or created itself is 
platform dependent.  Swap is pretty much the same thing as paging on 
Windows, and so \code{Sys.swapinfo()} is aliased as \code{Sys.pageinfo()}.  The 
two functions are identical within a given platform.

All operating systems basically treat swap/page files the same.  There are minor 
differences in allocation; for example, Mac OS X bizarrely uses unused space on 
the \code{/boot} partition for swap.  But the basic principle is the same: dump 
stuff to disk.



\subsection{Sys.meminfo() and Sys.swapinfo()}

%%% FIXME
As of \pkg{memuse}~$\geq 2.0$, we include the \code{Sys.meminfo()} and 
\code{Sys.swapinfo()} utilities.  The former gathers some minimal 
hardware information about available memory, and the latter does the same but 
for swap.

To use, simply call:
\begin{lstlisting}
Sys.meminfo()
Sys.swapinfo()
\end{lstlisting}

And a list will be returned containing some information.  By default, this will 
only contain a simplified profile of total and ``available'' (more on this 
later) physical memory. 





\subsubsection{Linux}
In Linux, the total available ram can be broken down into
\begin{itemize}
  \item truly unused (freeram)
  \item used for buffer cache (bufferram)
  \item used for file caching (cachedram)
  \item used by programs
\end{itemize}

By default, memuse will collapse the first three into \code{freeram}, because that's how most people think about it; and if you open up some kind of system monitor program, they do the same thing.  By \emph{truly unused}, we mean that nothing on the system has made any claim to it (according to the kernel\dots).  The memory allocated for buffers is .

For more details, \href{https://www.redhat.com/advice/tips/meminfo.html}{Redhat} has a pretty good breakdown of the various items.


These values should agree with (and in some cases, are taken directly from) those found in 
\code{/proc/meminfo}, with a small sample of the file provided:

\begin{center}
\begin{minipage}{.4\textwidth}
\begin{Output}
MemTotal:        8053556 kB
MemFree:         1761144 kB
Buffers:          808096 kB
Cached:          2327648 kB
SwapCached:            8 kB
\end{Output}
\end{minipage}
\end{center}

The actual implementation of \code{meminfo()} uses \code{sysinfo} (see \code{man 2 sysinfo}) for everything but cached memory, which is read directly from \code{/proc/meminfo}.


\subsubsection{Windows}


\subsubsection{Mac OS X}





\subsubsection{FreeBSD}

For some reason, FreeBSD doesn't support \code{_SC_AVPHYS_PAGES} in \code{sysconf} like Solaris and Linux do.  So for getting free memory, I have to make a call to \code{sysctlbyname}.

Also, the only way I could see to support getting free swap on FreeBSD is to make libkvm a dependency, 
which is kind of ridiculous.  Also, I figure you folks will just call \code{sysctl(8)} anyway and 
wonder why I even went to all this trouble in the first place, so you get what you get.



\subsubsection{Miscellaneous *NIX Variants}

Everybody seems to have their own little wiggle to \code{sysctl} for getting 
various memory statistics, and I just don't have the kind of time or resources 
to set up 100 different BSD-variant vm's to test that the man pages aren't lying 
to me.  As such, all *NIX's outside of Linux and Mac get only total ram and 
``free ram'', as reported by \code{sysconf}.

\code{Sys.swapinfo()} is not supported on these platforms at this time.




\subsection{Sys.procmem()}

We also provide a utility for discovering the amount of ram being used by the 
current \R process.  To use it, simply call:
\begin{lstlisting}[language=rr]
Sys.procmem()
\end{lstlisting}

This is utility is only supported on Linux, Windows, and Mac.  On Linux and 
Windows, the utility will also return the maximum amount of ram used by the 
current \R process.  You can see how this works by calling:

\begin{lstlisting}[language=rr]
library(memuse)

Sys.procmem()
x <- rnorm(1e7)
object.size(x)
Sys.procmem()
rm(x)
gc(FALSE)
Sys.procmem()
\end{lstlisting}

Just as an example, on my machine, this produces:
\vspace{-.6cm}
\begin{Output}
> Sys.procmem()
$size
134.488 MiB

$peak
134.488 MiB

> x <- rnorm(1e7)

> object.size(x)
76.294 MiB

> Sys.procmem()
$size
210.785 MiB

$peak
210.785 MiB

> rm(x)
> gc(FALSE)
         used (Mb) gc trigger (Mb) max used (Mb)
Ncells 324437 17.4     597831 32.0   407500 21.8
Vcells 578176  4.5    9599428 73.3 10582203 80.8

> Sys.procmem()
$size
134.488 MiB

$peak
210.785 MiB

\end{Output}
