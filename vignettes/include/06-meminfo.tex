\section{Hardware Memory Information: meminfo()}

As of \pkg{memuse}~$\geq 2.0$, we include the \code{meminfo()} utility.  This function gathers some minimal hardware information about available memory.  This is \emph{extremely} platform dependent, and so we discuss some of the major implementations available.  

The platform with the best support in the \pkg{memuse} package is Linux.  This is largely because getting this information in Linux is very simple --- and the data itself is very well-defined --- compared to other platforms.  So get your shit together, other OS devs.

For clarity, we define some terms here.

Swap is pretend ram that you probably shouldn't ever use; generally I would rather have a program crash than send me into a 40 hour swap death.

All operating systems basically treat swap/page files the same.  There are minor differences in allocation; for example, Mac OS X bizarrely uses unused space on the \code{/boot} partition for swap.  But the basic principle is the same: dump stuff to disk.

For the remainder, we will be discussing how various values for ram are calculated.



\subsection{Linux}
In Linux, the total available ram can be broken down into
\begin{itemize}
  \item truly unused (freeram)
  \item used for buffer cache (bufferram)
  \item used for file caching (cachedram)
  \item used by programs
\end{itemize}

By default, memuse will collapse the first three into \code{freeram}, because that's how most people think about it; and if you open up some kind of system monitor program, they do the same thing.  By \emph{truly unused}, we mean that nothing on the system has made any claim to it (according to the kernel\dots).  The memory allocated for buffers is .

For more details, \href{https://www.redhat.com/advice/tips/meminfo.html}{Redhat} has a pretty good breakdown of the various items.


These values can easily be found in \code{/proc/meminfo}, with a small sample of the file provided:

\begin{center}
\begin{minipage}{.4\textwidth}
\begin{Output}
MemTotal:        8053556 kB
MemFree:         1761144 kB
Buffers:          808096 kB
Cached:          2327648 kB
SwapCached:            8 kB
\end{Output}
\end{minipage}
\end{center}

The actual implementation of \code{meminfo()} uses \code{sysinfo} (see \code{man 2 sysinfo}) for everything but cached memory, which is read directly from \code{/proc/meminfo}.





\subsection{Mac OS X}






\subsection{Windows}

The \href{http://msdn.microsoft.com/en-us/library/windows/desktop/aa366589%28v=vs.85%29.aspx}{Windows documentation} 
on this is surprisingly clear and thorough.  Well done on finally getting something right, Microsoft.






\subsection{Miscellaneous *NIX Variants}

Everybody seems to have their own little wiggle for to \code{sysctl} for getting various memory statistics, and I just don't have the kind of time or resources to set up 100 different BSD-variant vm's to test that the man pages aren't lying to me.  As such, all *NIX's outside of Linux and Mac get only total ram and ``free ram'', as reported by \code{sysconf}.

Contributions are welcome.

