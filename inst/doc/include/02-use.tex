


\section{Using the MemUse Package}

\subsection{Size Matters, and How You Are Using It is Wrong}
The core of the \pkg{MemUse} package is the \code{memuse} class object.  You can construct a \code{memuse} object via the \code{memuse()} or \code{mu()} constructor.  The constructor has several options.  You can pass the size of the object, the unit, the unit prefix (IEC or SI), and the unit names (short or long).  The size is the number of bytes, scaled by some factor depending on the unit.  The unit is an abstract rescaling unit, like percent, used for the sake of simple comprehension at larger scales; for example, kilobyte and kibibyte are the typical storage units to represent ``roughly a thousand'' bytes (more on this later).  Finally, the unit names are for printing, i.e., controlling whether the long version (e.g., kilobyte) or short version (kB) is used.
\begin{table}[ht]
  \centering
  \begin{tabular}{|lll|lll|}\hline
    \multicolumn{3}{|c}{IEC Prefix} & \multicolumn{3}{|c|}{SI Prefix} \\\hline
    Short & Long & Factor & Short & Long & Factor\\\hline
    b & byte & 1 & b & byte & 1\\
    KiB & kibibyte & $2^{10}$ & kB & kilobyte & $10^3$\\
    MiB & mebibyte & $2^{20}$ & MB & megabyte & $10^6$\\
    GiB & gibibyte & $2^{30}$ & GB & gigabyte & $10^9$\\
    TiB & tebibyte & $2^{40}$ & TB & terabyte & $10^{12}$\\
    PiB & pebibyte & $2^{50}$ & PB & petabyte & $10^{15}$\\
    EiB & exbibyte & $2^{60}$ & EB & exabyte & $10^{18}$\\
    ZiB & zebibyte & $2^{70}$ & ZB & zettabyte & $10^{21}$\\
    YiB & yobibyte & $2^{80}$ & YB & yottabyte & $10^{24}$\\\hline
  \end{tabular}
  \caption{Units, Unit Prefices, and Scaling Factors for Byte Storage}
  \label{tab:units}
\end{table}
Table~\ref{tab:units} gives a complete list of the different units for the different prefices.

So for example, 1 kilobyte (kB) is equal to 1000 bytes, but 1 kibibyte (KiB) is equal to 1024 bytes.  And so 1 kB is roughly $0.977$ KiB.

The reason for this odd distinction is that there is general ambiguity in the public versus technical definition of these terms.  People, even those who know the difference (myself included) almost overwhelmingly use, for example, gigabyte when they mean gibibyte.  The reason for this is obvious; ``gibibyte'' sounds fucking stupid.  This actually gets all the more confusing because in addition to conflating 1 MB with 1 MiB, ISP's advertise their speeds in terms of bits\footnote{1 byte is 8 bits} \emph{using the same goddamn symbol}, because they're huge assholes.

Another example is when people talk about \twiddle{big data}.  Often I/O people will use the term ``terabytes'' or ``exabytes'' and mean it.  Rescaling these units into the ones people are generally more familiar with is simple with the \code{MemUse} package:
\begin{lstlisting}[language=rr]
> swap.prefix(mu(size=1, unit="tb", unit.prefix="SI"))
0.909 TiB
> swap.prefix(mu(size=1, unit="pb", unit.prefix="SI"))
0.888 PiB
\end{lstlisting}

These sizes represent an impressive amount of data, but this ambiguity in naming conventions allows people to lie a bit.  For all of these reasons, since the package is meant to be useful for understanding R object size, the default behavior is somewhat complicated, but can be summarized as trying to provide what most people meant in the first place.  We achieve this by offering several default string objects which the user can easily control.  These units are \code{.UNIT}, \code{.PREFIX}, and \code{.NAMES}.  



\subsection{Default Parameters}

The \code{.UNIT} object defaults to \code{best} and should probably just be left alone.  Functions that need to know an input unit, such as the constructor, have default argument \code{unit=.UNIT}.  Realistically, you are probably better off modifying that argument as necessary than changing \code{.UNIT}.  For example, you want to construct a 100 KiB \code{memuse} object, you probably just want to call
\begin{lstlisting}[language=rr]
mu(100, "KiB")
\end{lstlisting}
This is equivalent to calling
\begin{lstlisting}[language=rr]
mu(102400)
\end{lstlisting}
since the default \code{.UNIT=best} will make the choice to switch the units from b to KiB once you breach 1024 bytes.  This sounds a lot more confusing than it really is.

More useful is the \code{.PREFIX} parameter.  This must either be \code{SI} or \code{IEC}, with the latter being the package default.  
\begin{lstlisting}[language=rr]
> .PREFIX <- "SI"
> x <- mu(10, "kb")
> x
10.000 KB
> swap.prefix(x)
9.766 KiB
\end{lstlisting}



\subsection{Methods}
Aside from the constructor, you have already seen one very useful method:  \code{swap.prefix()}.  In addition to these, we have several other obvious methods, such as \code{swap.unit()}, \code{swap.names()}, \code{print()}, \code{show()}, etc.  But we also have some simple arithmetic, namely \code{`+`} (addition), \code{`*`} (multiplication), and \code{`\^{}`} (exponentiation).  So for example:
\begin{lstlisting}[language=rr]
> mu(100) + mu(200)
300.000 B
> mu(100) * mu(200) # 100*200/1024
19.531 KiB
\end{lstlisting}
It's not hard to implement other things like division, but I didn't because I thought it was stupid.

Finally, we have the methods that inspired the creation of this entire dumb thing in the first place:  \code{howbig()} and \code{howmany()}.  The former takes in the dimensions of a matrix (\code{nrow} rows and \code{ncol} columns) and returns the memory usage (as the package namesake would imply) of the object.  So for example, if you wanted to perform a principal components decomposition on a 100,000 by 100,000 matrix via SVD (as we have), then you would need:
\begin{lstlisting}
> howbig(100000, 100000)
74.506 GiB
\end{lstlisting}
Of ram just to store the data.  Another interesting anecdote about this sized matrix is that we were able to generate it in just over a tenth of a second.  Pretty cool, eh?

As mentioned before, there is also the \code{howmany()} method which does somewhat the reverse of \code{howbig()}.  Here you pass a \code{memuse} object and get a matrix size out.  You can pass (exactly) one argument \code{nrow} or \code{ncol} in addition to the \code{memuse} object; the method will determine the maximum possible size of the outlying dimension in the obvious way.  If no additional argument is passed, then the largest square matrix dimensions will be returned.



\subsection{Package Demos}

In addition to all of the above, the \pkg{MemUse} package includes several demos.  You can execute them via the command:
\begin{lstlisting}[title=List of Demos]
### (Use Rscript.exe for windows systems)

# Basic construction/use of memuse objects
Rscript -e "demo(demo, package='MemUse', ask=F, echo=F)"
# Arithmetic
Rscript -e "demo(demo2, package='MemUse', ask=F, echo=F)"
# howbig/howmany examples
Rscript -e "demo(demo3, package='MemUse', ask=F, echo=F)"
\end{lstlisting}



\subsection{Comparison to \code{object.size()}}

\proglang{R} contains a handy tool for telling you how big an already allocated object is, the \code{object.size()} method.  This package is effectively an extension of that method for un-allocated objects, provided your objects are numeric (more on this later).

So say we have the vector \code{x <- 1.0}.  This should be using 8 bytes to store that \code{1.0} as a double, right?  Well\dots
\begin{lstlisting}[language=rr]
> object.size(1.0)
48 bytes
\end{lstlisting}

So where is all that extra space coming from?  Simply put, \proglang{R} objects are more than just their data.  They contain a great deal of very useful metadata, which is where all the nice abstraction comes from.  Whenever you create a vector, \proglang{R} keeps track, for example, its length.  If you do not appreciate this convenience, go learn \proglang{C} and then get back to me.  

For vectors, this overhead is 40 bytes, regardless of the type of data.  Matrices, unsurprisingly cost more, clocking in at 200 bytes overhead.  It is worth noting that this overhead does not scale; it is on a per-object basis.  So we don't need 40 bytes for each element of a vector when just 8 would do (in the case of double precision values).  We need 40 plus 8 per element:
\begin{lstlisting}[language=rr]
> # 2 elements
> 40+8*2
[1] 56
> object.size(rnorm(2))
56 bytes
> # 100.000 elements
> 40+1e5*8
[1] 800040
`> object.size(rnorm(1e5))
800040 bytes
\end{lstlisting}

The story is slightly more complicated for integer data (and a lot more complicated for strings; see the following section).  On my machine (and probably yours, but not necessarily), ints costs 4 bytes.  However, \proglang{R} does some aggressive allocation, most likely for reasons of efficiency:
\begin{lstlisting}[language=rr]
> object.size(1L:3L)
56 bytes
> object.size(1L:4L)
56 bytes
\end{lstlisting}
Here we see \proglang{R} allocating more bytes than it needs for integer vectors sometimes, choosing to allocate in 16 byte chunks rather than 8 byte chunks.

The \pkg{MemUse} package does not adjust for this overhead, because it honestly just doesn't matter.  This overhead is really not worth worrying about, and when you think about all the abstraction it buys you, it's a hell of a bargain.  If you have a million R objects stored, you're wasting less than a mebibyte ($1024^2$ bytes); so you would need a billion objects to use just about a gibibyte ($1024^3$ bytes) on overhead.  And if you're doing that kind of silly nonsense, my advice would be to learn how to properly use data structures.



\subsection{Strings}

String objects have been avoided up until this point because they are much more difficult to describe in general, unless they have a great deal of regularity imposed on them.  In \proglang{R}, strings by default are allocated to use 56 bytes (not counting overhead), unless they need more.  I'm not sure why this value was chosen, but 56 byte strings will allow for the storage of 7 chars (like \code{a} but not \code{aa}).  Each char costs 1 byte, so there's some fat overhead for the strings here, and almost certainly an additional byte held out for the null terminator.  So for example, recall that a vector allocates 40 bytes of overhead, so the vector string \code{letters} should use $56\times 26 + 40$ bytes.  We can easily verify that this is the case:
\begin{lstlisting}[language=rr]
> 56*26+40
[1] 1496
> object.size(letters)
1496 bytes
\end{lstlisting}

If you have a string with more than 7 chars, \proglang{R} will allocate extra space in 8-16 byte blocks.  After the initial 8 byte allocation (7 chars $+$ null terminator), if you need more you get an additional 8 bytes (in reality this is probably a contiguous 16 byte allocation; I have not bothered to check).  Beyond that, storage is allocate in 16 byte blocks for each string.  For example:
\begin{lstlisting}[language=rr]
>  object.size(c(paste(rep("a", 7), collapse=""), "a")) 
152 bytes
>  object.size(c(paste(rep("a", 7+1), collapse=""), "a")) 
160 bytes
>  object.size(c(paste(rep("a", 7+8+1), collapse=""), "a")) 
176 bytes
>  object.size(c(paste(rep("a", 7+8+16+1), collapse=""), "a")) 
192 bytes
\end{lstlisting}

If you have a vector of strings with them of varying lengths, the allocation of individual elements is handled on a case-by-case basis.  Consider the following:
\begin{lstlisting}[language=rr]
> object.size(c(paste(rep("a", 7+8+16+1), collapse=""), "a")) 
192 bytes
\end{lstlisting}

This object (the vector of 2 elements with first element ``aaaaaaaaaaaaaaaaaaaaaaaaaaaaaaaa'' and second element ``a'') is using 40 bytes for the vector, $56+8+16+16$ bytes for the first element, and 56 bytes for the second.

For all of these reasons, and given the fact that I almost never (ever) deal with character data, I have not bothered to make any attempt to extend, for example, \code{howmany()} or \code{howbig()}, to incorporate strings.  Deal with it, nerd.